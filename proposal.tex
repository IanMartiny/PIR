\documentclass{article}

\title{PIR}
\author{Ian Martiny}
\date{}
\begin{document}
\maketitle

\section{Introduction}
Client security is an increasingly important requirement of modern day tools.
In a world where much of our data is stored online great care must be given to
how data is stored. Recently there have been pushes to use encrypted databases.
They claim that by encrypting the requests and data passive adversaries are
unable to gain any information on either the query or the data returned. 

This encryption ranges from end-to-end encryption between the client and server
where the end data is still stored un-encrypted~\cite{mylar, arx}, to
client-to-client encryption where the server itself does not know how to decrypt
any information it is given~\cite{seabed,cryptdb}. However it has been 
noticed~\cite{morningPaper} that these schemes do not prevent attackers from
gaining information about queries and responses. Due to logging multiple
requests and responses attackers can glean information about similar requests,
even possibly reconstructing encrypted data.

In this project we propose to augment this idea of encrypted databases to use
private information retrieval (PIR)~\cite{PIR,CPIR}. PIR obscures information
from adversaries and servers by allowing clients to provide randomly encrypted
queries to the server that can then use these queries on their local storage to
return the desired content. While similar to ideas presented in Seabed and
CryptDB, computational PIR adds the benefit of randomized encryption, such that
no query can be distinguished from any other query.

\section{Contributions}
In this project we will implement a PIR library that will allow others to easily
add PIR security to their projects. Initially this will require a separate
program to handle client interactions with servers, however eventually our goal
is that this will be extended to allow online clients to simply make Javascript
queries. 

Additionally, a framework will be designed to allow clients to store and
retrieve information from an example server.

\bibliographystyle{unsrt}
\bibliography{refs}
\end{document}